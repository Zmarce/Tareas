\documentclass{beamer}
\usepackage[utf8]{inputenc}
\usepackage[T1]{fontenc}
\usepackage{helvet}

\title{Acoplamiento molecular entre la proteína E del SARS‐CoV‐2 y la amantadina}
\author{Alba Marcela Zárate Rochín}
\date{Enero 2021} 
\institute{Curso de Computación Científica y Bioinformática\\Doctorado en Investigaciones Cerebrales\\Universidad Veracruzana}

\begin{document}
\begin{frame}[plain,t]
\titlepage
\end{frame}


\begin{frame}
\frametitle{Antecedentes}

\begin{itemize}
\item SARS‐CoV‐2 virus responsable de la enfermedad COVID-19
	\begin{itemize}
	\item Infección del tracto respiratorio que puede causar neumonía y una viremia que afecta a otros tejidos distantes como el riñón, el corazón y el cerebro.\\
	\item Proceso hiperinflamatorio 
	\end{itemize}
\end{itemize}

\begin{figure}
    \centering
    \includegraphics[width=0.5\textwidth]{cov2.jpeg}
    \label{fig:cov2}
\end{figure}
    \end{frame}

\begin{frame}
\frametitle{}

\begin{itemize}
\item Amantadina
	\begin{description}
	\item Potente antiinflamatorio con capacidad neuroprotectora
	\end{description}
\end{itemize}

\begin{figure}
    \centering
    \includegraphics[width=0.75\textwidth]{amantadina.png}
    \caption{Estructura molecular}
    \label{fig:amantadina}
\end{figure}
    \end{frame}

\begin{frame}
\frametitle{}

\begin{itemize}
\item Amantadina
	\begin{description}
	\item Eficaz en la profilaxis y el tratamiento de las infecciones por influenza A\\
	Actividad antiviral selectiva\\
	Inhibe el inicio de la infección o el ensamblaje del virus 
	\end{description}
\end{itemize}

\begin{figure}
    \centering
    \includegraphics[width=0.5\textwidth]{influeza.jpg}
    \caption{Bloqueo del canal M2}
    \label{fig:influenza}
\end{figure}
    \end{frame}


\subsection{Text}
\begin{frame}
\frametitle{Plantemiento del problema}
\begin{itemize}
\item El virus del SARS-CoV-2 expresa un proteína de envoltura, conocida como proteína E, fundamental para la gemación viral y cuya actividad es inhibida por la amantadina.
\end{itemize}

\begin{figure}
    \centering
    \includegraphics[width=0.75\textwidth]{protE.jpg}
    \caption{Proteína E}
    \label{fig:protE}
\end{figure}
    \end{frame}


\begin{frame}
\begin{itemize}
\item Hipótesis
	\begin{itemize}
	\item La amantadina bloquea el canal de la proteína E evitando la liberación del núcleo viral al citoplasma celular.
	\item Necesidad de nuevos estudios enfocados en las propiedades de la amantadina y en su posible interacción con la proteína E para indagar sobre su utilidad en posibles tratamientos en el COVID-19.
	\end{itemize}
\item Objetivo
	\begin{description}
	\item Realizar un análisis del acoplamiento molecular entre la molécula de la amantadina y el canal de la proteína E del SARS-CoV-2. 
	\end{description}
\end{itemize}
\end{frame}

\begin{frame}
\frametitle{Metodología}
\begin{itemize}
\item Proteína E (Protein Data Bank) y Amantadina (Drug Bank) en formato PDB
\item Preparación de las moléculas
\item Programa AutoDock 4.2.6., ejecutable de AutoDock vina y visualización en AutoDock Tools
\item Generación de archivos con formato PDBQT
\item Delimitar zona de acoplamiento
\item Establecer parámetros de búsqueda, modos, exhaustividad y rango de energía 
\end{itemize}
\end{frame}

\begin{frame}
\frametitle{Resultados}

\begin{figure}
    \centering
    \includegraphics[width=1\textwidth]{Grid.png}
    \caption{Inserción de ligando y delimitación de la zona de acoplamiento}
    \label{fig:Grid}
\end{figure}
    \end{frame}

\begin{frame}
\begin{figure}
    \centering
    \includegraphics[width=1\textwidth]{Results1.png}
    \caption{Resultados según la energía de afinidad de las conformaciones encontradas}
    \label{fig:Results1}
\end{figure}
    \end{frame}

\begin{frame}
\begin{figure}
    \centering
    \includegraphics[width=1\textwidth]{Results.png}
    \caption{Resultado principal con energía de afinidad de -5.6 kcal/mol}
    \label{fig:Results}
\end{figure}
    \end{frame}

\begin{frame}
\frametitle{Conclusiones}
Entender los mecanismos de interacción entre la amantadina y la proteína E puede contribuir en la generación de tratamientos para esta enfermedad y otros usos clínicos asociados. En ese sentido, el uso de técnicas de mecánica molecular, como el docking, es muy útil para predecir energía y modos de enlace entre proteínas y ligandos. Este tipo de información es importante para el estudio de compuestos con potencial terapéutico. De manera general, se determinó la interacción in silico entre la proteína E y la amantadina con la finalidad de mostrar interacciones específicas para futuros análisis. 

\end{frame}

\end{document}